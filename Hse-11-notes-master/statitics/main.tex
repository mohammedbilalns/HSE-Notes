\documentclass[12pt]{article}
\usepackage{graphicx} % Required for inserting images
\usepackage[margin=2cm]{geometry}
\usepackage{multicol,amsmath, amssymb}
\usepackage{xcolor}
\usepackage{titlesec}
\usepackage{pgfplots}

\titleformat{\subsection}
{\color{red}\normalfont\Large\bfseries}
{\thesubsection}{1em}{}

\titleformat{\subsubsection}
{\color{blue}\normalfont\large\bfseries}
{\thesubsubsection}{1em}{}

\titlespacing{\subsection}{0pt}{0pt}{0pt} % Adjust the spacing here
\titlespacing{\subsubsection}{0pt}{\baselineskip}{0pt} % Adjust the spacing here
\newcommand{\summation}[2]{\sum\limits^{#1}_{#2}}

\usepackage{fancyhdr}
\usepackage{hyperref} % For hyperlinks

% Define the URL for the footer
\newcommand{\myURL}{https://mohammedbilalns.github.io/Math-Demystified/}

% Set up fancy headers and footers
\pagestyle{fancy}
\fancyhf{} % Clear header and footer
\rfoot{\href{\myURL}{\myURL}} % Set the right part of the footer as a hyperlink

\begin{document}
\begin{center}
    {\LARGE \textbf{Statitics} }
\end{center}

\begin{multicols}{2}

\textbf{\Large Measures of Dispersion}

This gives a measure of the dispersion of the observation around the measure of central tendency of the data collected.

\subsection*{Range}
Range of a data = Maximum value – Minimum value.

\subsection*{Mean Deviation}
Mean deviation of central tendancy 'a' of a data is given by 
$$M.D (a) = \frac{\text{Sum of absolute values of deviations from 'a'}}{\text{Number of observations}}$$

\begin{itemize}
    \item \subsubsection*{Ungrouped data} Mean deviation for the data  $x_1, x_2, x_3, ...., x_n$ is  
    $$M.D(a)=\frac{ \summation{n}{i=1} |x_i -a| }{n}$$
    \begin{itemize}
        \item $Mean(\bar{x}) =\frac{\summation{n}{i=1}x_i}{n}$
        \item Median is $\frac{n+1}{2} ^{th}$ term  when n is odd ,mean of $\frac{n}{2} ^{th}$ and $\frac{n}{2}+1 ^{th}$ term when n is even
    \end{itemize}

    \item \subsubsection*{Grouped Discrete frequency distribution } Mean deviation for the data    $x_1, x_2, ..., x_n$ occurring with frequencies $f_1, f_2 , ..., f_n$ respectively is 
    $$ M.D(a)=\frac{ \summation{n}{i=1} f_i|x_i -a| }{ \summation{n}{i=1}f_i}$$

    \begin{itemize}
        \item $Mean(\bar{x}) =\frac{\summation{n}{i=1}f_i x_i}{\summation{n}{i=1}f_i}$
        \item To find the median form a column for cumulative frequencies
    \end{itemize}

    \item \subsubsection*{Grouped Continuous frequency distribution }A continuous frequency distribution is a series
    in which the data are classified into different class-intervals without gaps alongwith
    their respective frequencies.
    
    To find the Mean deviation about a central tendancy 'a' , choose $x_i$ as midpoint of the intervals and convert it into Discrete.

    \begin{itemize}
        \item $Mean(\bar{x}) =\frac{\summation{n}{i=1}f_i x_i}{\summation{n}{i=1}f_i}$
        \item median =$l+\frac{c(\frac{N}{2}-f_0)}{f_1}$\\
        $l$ -lower limit of the median class \\
        $f_0 $- Cumulative frequency of the class preceding the median class.\\
        $f_1$ - Cumulative frequency of the median class.\\
        $c$ - width of the interval \\
        $N-\summation{i=0}{n}f_i$



    \end{itemize}

    \textbf{Shortcut method for finding the mean of frequency distribution }

    \begin{enumerate}
        \item Select an assumed mean 'a' (value close to middle)
        \item substract assumed mean from eaach $x_i$
        \item choose appropriate h and divide each $x_i -a$ by h and define it as $d_i$
        \item find mean using the fromula 
        $$\bar{x}= a+\frac{\summation{i=1}{n}f_i d_i}{N} \times h$$
    \end{enumerate}

    \subsection*{Variance and Standard Deviation}
standard deviation$ = \sqrt{variance}$
    \subsubsection*{Ungrouped Data}
    variance  for the data  $x_1, x_2, x_3, ...., x_n$ is  
    $\sigma^2 = \frac{1}{n} \summation{i=1}{n} (x_i-\bar{x})^2$

    \subsubsection*{Grouped  frequency distribution } Variance for the data    $x_1, x_2, ..., x_n$ occurring with frequencies $f_1, f_2 , ..., f_n$ respectively is 
    $$\sigma^2=\frac{1}{N} \summation{i=1}{n}f_i(x_i-\bar{x})^2$$

    \textbf{Another formula for standard deviation}
    $$\sigma = \frac{1}{N} \sqrt{N \summation{i=1}{n}f_i x_i^2 -(\summation{i=1}{n}f_i x_i)^2}$$

    \textbf{Shortcut method to find variance and standard deviation}

    Let the assumed mean be ‘A’ and the scale be reduced to $\frac{1}{h}$ times (h being the
width of class-intervals). Let the step-deviations or the new values be $y_i$.
$$y_i =\frac{x_i- A}{h}$$
then $$\sigma =\frac{h}{N} \sqrt{N \summation{i=1}{n}f_i y_i^2-(\summation{i=1}{n}f_i y_i)^2}$$
\end{itemize}










\end{multicols}
\end{document}