\documentclass[12pt]{article}
\usepackage{graphicx} % Required for inserting images
\usepackage[margin=2cm]{geometry}
\usepackage{multicol,amsmath, amssymb}
\usepackage{xcolor}
\usepackage{titlesec}
\usepackage{pgfplots}

\titleformat{\subsection}
{\color{red}\normalfont\Large\bfseries}
{\thesubsection}{1em}{}

\titleformat{\subsubsection}
{\color{blue}\normalfont\large\bfseries}
{\thesubsubsection}{1em}{}

\titlespacing{\subsection}{0pt}{0pt}{0pt} % Adjust the spacing here
\titlespacing{\subsubsection}{0pt}{\baselineskip}{0pt} % Adjust the spacing here
\newcommand{\summation}[2]{\sum\limits^{#1}_{#2}}
\begin{document}
\begin{center}
    {\LARGE \textbf{Sets,Relations and functions} }
\end{center}
\begin{enumerate}
    \item If $A = [(x, y) : x^2 + y^2 = 25]$ and $B = [(x, y) : x^2 + 9y^2 = 144]$, then $A \cap B$ contains ------ points.
    \item  In a college of 300 students, every student reads 5 newspapers, and every newspaper is read by 60 students. The number of newspapers is -----
    \item  Let R be the relation on the set R of all real numbers defined by $a R b$ if and only if $|a - b| \leq 1$. Then R is
     \begin{enumerate}
        \item Reflexive Symmetric and Transitive
        \item Not Reflexive But Symmetric and Transitive
        \item Reflexive and Symmetric but not Transitive
    \end{enumerate}
    \item if $f(x)=\frac{x-3}{x+1}$ then $f(f(f(x)))$ is 
     \begin{enumerate}
        \item $\frac{1}{x}$
        \item $3x$
        \item $x$
        \item $\frac{1+x}{x}$
    \end{enumerate}
    \item  If $f (x) = 3x - 5$, then $f^{-1}(x)$ is?
    \item $f(x)=sin^{-1}(4-(x-7)^3)^{\frac{1}{7}}$ then $f^{-1}(x)=$------?
    \item If $f:R \rightarrow R$, $f(x)=x^2+1$ then $f^{-1}(2) \cup f^{-1}(17)=$ ------?
    \item If function $f$ satisfies the equation $3f(x)+2f(\frac{x+59}{x-1})=10x+30, x \not = 1$ then $f(7)=$ ------?
    \item $g:R \rightarrow R,$ $g(x)=3+\sqrt[3]{x}$ and $f(g(x))=2-\sqrt[3]{x}+x$ then $f(x)=$ ------?
    \item The domain of the function $f(x)=\frac{1}{\sqrt{|x|-x}}$ is 
    \begin{enumerate}
        \item $(-\infty,\infty)$
        \item $(0,\infty)$
        \item $(-\infty, 0)$
        \item $(-\infty,\infty)-\{0\}$
    \end{enumerate}
   
\end{enumerate}







\end{document}
