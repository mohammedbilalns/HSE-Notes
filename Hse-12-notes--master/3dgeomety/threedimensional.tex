\documentclass[12pt]{article}
\usepackage{graphicx} % Required for inserting images
\usepackage[margin=2cm]{geometry}
\usepackage{multicol,amsmath, amssymb}
\usepackage{xcolor}
\usepackage{titlesec}
\usepackage{pgfplots}
\usepackage{tikz}

\titleformat{\subsection}
{\color{red}\normalfont\Large\bfseries}
{\thesubsection}{1em}{}

\titleformat{\subsubsection}
{\color{blue}\normalfont\large\bfseries}
{\thesubsubsection}{1em}{}

\titlespacing{\subsection}{0pt}{0pt}{0pt} % Adjust the spacing here
\titlespacing{\subsubsection}{0pt}{\baselineskip}{0pt} % Adjust the spacing here
\newcommand{\summation}[2]{\sum\limits^{#1}_{#2}}
\usepackage{fancyhdr}
\usepackage{hyperref} % For hyperlinks

% Define the URL for the footer
\newcommand{\myURL}{https://mohammedbilalns.github.io/Math-Demystified/}

% Set up fancy headers and footers
\pagestyle{fancy}
\fancyhf{} % Clear header and footer
\rfoot{\href{\myURL}{\myURL}} % Set the right part of the footer as a hyperlink

\begin{document}
\begin{center}
    {\LARGE \textbf{Three Dimensional Geometry} }
\end{center}
To refer a point in space we require a third axis (say z-axis) which leads to the concept of three-dimensional geometry. In this chapter we study the concept of direction cosines, direction ratios, equation of a line and a plane, angle between two lines and two planes, angle between a line and a plane, shortest distance between two skew lines, distance of a point from a plane

\subsection*{Direction Cosines and direction ratios}

Consider a directed line passing through the origin makes angles $\alpha, \beta$ and $\gamma$ with the positive
direction x-axis, y-axis, and z-axis. Then $\alpha, \beta,$ and $\gamma$ are called direction angles. The cosine of $\alpha, \beta,$ and $\gamma$ are called direction cosines. Generally $cos \alpha = l$, $cos \beta = m$ and $cos \gamma = n$ . Any scalar multiple of direction cosines are called direction ratios.

\begin{itemize}
    \item If (a, b, c) is the coordinate of a point P then a,b,c is a direction ratio of the directed line passing along P and origin. Direction cosines will be $$l=\frac{a}{\sqrt{a^2+b^2+c^2}}, m=\frac{b}{\sqrt{a^2+b^2+c^2}},n=\frac{c}{\sqrt{a^2+b^2+c^2}}$$
    \item $l^2+m^2+n^2=1$ 
\end{itemize}
\subsubsection*{Direction cosines of a line passing through two points}
Direction ratios of a line segmnent passing through two points $P(x_1,y_1,z_1)$ and $Q(x_2,y_2,z_2)$  is $$x_2-x_1,y_2-y_1,z_2-z_1$$ Direction angles are $$\frac{x_2-x_1}{AB},\frac{y_2-y_1}{AB},\frac{z_2-z_1}{AB}$$

\subsection*{Equation of a line in space}

\subsubsection*{ Equation of a line through a given point $\vec{a}$ and parallel to given vector $\vec{b}$}
Let $\vec{a}$ be the position vector of the given point A with respect to the origin O of the coordinate system.Let l be the line which passes through the point A and is parallel to a given vector $\vec{b}$. Let $\vec{r}$ be the position vector of  an arbitary point P on the line then the equation of the line is $$\vec{r}=\vec{a}+\lambda \vec{b}$$ where $\lambda$ be any be any real number.
\subsubsection*{Cartesian form}
Let $(x_1,y_1,z_1)$ be coordinates of a given point in line and $a,b$ and $c$ be the direction ratios of the line then $$\frac{x-x_1}{a}=\frac{y-y_1}{b}=\frac{z-z_1}{c}$$ This is the \textbf{Cartesian} equation of the line

\subsection*{Angle between two lines}
Let $L_1$ and $L_2$ be two lines passing through the origin and with direction ratios $a_1,b_1,c_1$ and $a_2,b_2,c_2$ respectively. then the angle between them is $$cos \theta =\left|\frac{a_1a_2+b_1b_2+c_1c_2}{\sqrt{a_1^2+b_1^2+c_1^2}\sqrt{a_2^2+b_2^2+c_2^2}}\right|$$

Two lines with direction ratios $a_1,b_1,c_1$ and $a_2,b_2,c_2$ are 
\begin{itemize}
    \item perpedicular if $$a_1a_2+b_1b_2+c_1c_2=0$$
    \item parallel if $$\frac{a_1}{a_2}=\frac{b_1}{b_2}=\frac{c_1}{c_2}$$
\end{itemize}
\subsection*{Shortest distance between two lines}
If two lines in space intersect at a point,
then the shortest distance between them is
zero. Also, if two lines in space are parallel,
then the shortest distance between them
will be the perpendicular distance.

Furthermore there are lines which are neither intersecting nor parallel, such pair of lines are non coplanar and
are called \textbf{skew lines}.
\subsubsection*{Distance between two skew lines }

Let $l_1$ and $l_2$ be two skew lines with equations 
$$\vec{r}=\vec{a_1}+\lambda \vec{b_1},\vec{r}=\vec{a_2}+\lambda \vec{b_2}$$

Then the distance between them will be $$d=\left|\frac{(\vec{b_1}\times \vec{b_2}). (\vec{a_2} -\vec{a_1})}{|\vec{b_1} \times \vec{b_2}|}\right|$$
\end{document}
