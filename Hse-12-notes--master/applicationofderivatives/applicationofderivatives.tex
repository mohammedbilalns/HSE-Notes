\documentclass[12pt]{article}
\usepackage{graphicx} % Required for inserting images
\usepackage[margin=2cm]{geometry}
\usepackage{multicol,amsmath, amssymb}
\usepackage{xcolor}
\usepackage{titlesec}
\usepackage{pgfplots}

\titleformat{\subsection}
{\color{red}\normalfont\Large\bfseries}
{\thesubsection}{1em}{}

\titleformat{\subsubsection}
{\color{blue}\normalfont\large\bfseries}
{\thesubsubsection}{1em}{}

\titlespacing{\subsection}{0pt}{0pt}{0pt} % Adjust the spacing here
\titlespacing{\subsubsection}{0pt}{\baselineskip}{0pt} % Adjust the spacing here
\newcommand{\summation}[2]{\sum\limits^{#1}_{#2}}

\usepackage{fancyhdr}
\usepackage{hyperref} % For hyperlinks

% Define the URL for the footer
\newcommand{\myURL}{https://mohammedbilalns.github.io/Math-Demystified/}

% Set up fancy headers and footers
\pagestyle{fancy}
\fancyhf{} % Clear header and footer
\rfoot{\href{\myURL}{\myURL}} % Set the right part of the footer as a hyperlink

\begin{document}
\begin{center}
    {\LARGE \textbf{Application of Derivatives} }
\end{center}


    In this chapter we analyses the physical and geometrical applications of derivatives in real life such as to determine rate of change, to find tangents and normal to a curve, to find turning points, intervals in which the curve is increasing and decreasing, to find approximate value of certain quantities.
    \subsection*{Rate of Change}
    By $\frac{dy}{dx}$,we mean the rate of change of $y$ with respect to $x$. If $s$ is the displacement function in terms of $t$ and $v$ the velocity at that time. Then $$velocity\, = \frac{ds}{dt}$$  $$acceleration = \frac{dv}{dt}=\frac{d^2 s}{dt^2}$$

    \subsection*{Increasing and Decreasing Functions}
    Let $I$ be an interval contained in the domain of a real valued function $f$.
Then $f$ is said to be \begin{enumerate}
    \item  \textbf{increasing} on $I$ if $x_1 < x_2$ in $I \Rightarrow f(x_1) \leq f(x_2)$ for all $x_1, x_2 \in I$.
    \item \textbf{strictly increasing}  on $I$ if $x_1 < x_2$ in $I \Rightarrow f(x_1) < f(x_2)$ for all $x_1, x_2 \in I$.
    \item  \textbf{decreasing} on $I$ if $x_1 < x_2$ in $I \Rightarrow f(x_1) \geq f(x_2)$ for all $x_1, x_2 \in I$.
    \item  \textbf{strictly decreasing} on $I$ if $x_1 < x_2$ in $I \Rightarrow f(x_1) > f(x_2)$ for all $x_1, x_2 \in I$.

    \item  \textbf{constant} on $I$, if $f(x) = c$ for all $x \in I$, where $c$ is a constant.
    
\end{enumerate}
Let $x_0$ be a point in the domain of definition of a real valued function $f$.
    Then $f$ is said to be increasing, decreasing at $x_0$ if there exists an open interval $I$
    containing $x_0$ such that $f$ is increasing, decreasing, respectively, in $I$.

    \subsubsection*{Theorem 1}
    Let f be continuous on $[a, b]$ and differentiable on the open interval
$(a,b)$. Then \begin{enumerate}
    \item $f$ is increasing in $[a,b]$ if $f'(x) > 0$ for each $x \in(a, b)$
    \item $f$ is decreasing in $[a,b]$ if $f'(x) < 0$ for each $x \in(a, b)$
    \item $f$ is constant in $[a,b]$ if $f'(x) = 0$ for each $x \in(a, b)$


\end{enumerate}

\subsection*{Maxima and Minima}
Let $f$ be a function defined on an interval $I$. Then \begin{enumerate}
    \item f is said to have a \textbf{maximum value} in $I$, if there exists a point $c$ in $I$ such that
    $f (c) > f ( x )$ , for all $x \in I$

    The number $f (c)$ is called the \textbf{maximum value} of $f$ in $I$ and the point $c$ is called a
\textbf{point of maximum value} of f in I.

    \item $f$ is said to have a \textbf{minimum value} in $I$, if there exists a point $c$ in $I$ such that
    $f (c) < f (x)$, for all $x \in I$.

    The number $f (c)$,is called the \textbf{minimum value} of $f$ in $I$ and the point
$c$, is called a \textbf{point of minimum value} of f in I.
\item $f$ is said to have an \textbf{extreme value} in $I$ if there exists a point $c$ in $I$ such that
$f (c)$ is either a maximum value or a minimum value of $f$ in $I$.

The number $f (c)$, in this case, is called an extreme value of $f$ in $I$ and the point $c$
is called an \textbf{extreme point}.



\end{enumerate}
\subsubsection*{Local Minma and Local Maxima }
Let $f$ be a real valued function and let $c$ be an interior point in the domain
of $f$. Then
\begin{itemize}
    \item $c$ is called a point of \textbf{local maxima} if there is an $h > 0$ such that $$ f(c) \geq f(x) \, \text{for all $x$ in } (c-h,c+h),x \not = c$$ The value of $f(c)$ is called \textbf{local maximum} value of $f$
    \item $c$ is called a point of \textbf{local minima} if there is an $h > 0$ such that $$ f(c) \leq f(x) \, \text{for all $x$ in } (c-h,c+h),x \not = c$$ The value of $f(c)$ is called \textbf{local minimum} value of $f$

\end{itemize}
\subsubsection*{Theorem 2}
Let $f$ be a function defined on an open interval $I$. Suppose $c \in I$ be any
point. If $f$ has a local maxima or a local minima at $x = c$, then either $f'(c) = 0$ or $f$ is not
differentiable at $c$.

\subsubsection*{Theorem 3 (First Derivative Test)}
\begin{enumerate}
    \item If $f'(c) = 0$ and $f'(x)$ changes its sign from positive to negative from left to right of $x = c$, then the point is a local maximum point.
    \item If $f'(c) = 0 $ and $f'(x)$ changes its sign from negative to positive from left to right of $x = c$, then the point is a local minimum point.
    \item If $f'(c) = 0$ and if there is no change of sign for $f'(x)$ from left to right of $x = c$, then the point is a inflexion point.
\end{enumerate}
\subsubsection*{Theorem 4 (Second Derivative Test)}
Let $f$ be a function defined on an interval $I$
and $c \in I$. Let $f$ be twice differentiable at $c$. Then
\begin{enumerate}
    \item If $f'(c) = 0$ and $f”(c) < 0$ , then $x = c$ is a local maximum point.
    \item If $f'(c) = 0$ and $f”(c) > 0$, then $x = c$ is a local minimum point.
    \item If $f'(c) = 0$ and $f”(c) = 0$, then the test fails and go to first derivate test for checking maxima and minima.
\end{enumerate}

\subsubsection*{Maximum and Minimum Values of a Function in a Closed Interval}

Consider $f(x)=x+1,x \in (0,1)$ .the function is continuous on (0, 1) and neither has a maximum value
nor has a minimum value. Further, we may note that the function even has neither a
local maximum value nor a local minimum value.

However, if we extend the domain of $f$ to the closed interval $[0, 1]$, then f still may
not have a local maximum (minimum) values but it certainly does have maximum value
$3 = f (1)$ and minimum value $2 = f (0)$. The maximum value $3$ of $f$ at $x = 1$ is called
\textbf{absolute maximum value} (global maximum or greatest value) of $f$ on the interval
$[0, 1]$. Similarly, the minimum value $2$ of $f$ at $x = 0$ is called the \textbf{absolute minimum
value} (global minimum or least value) of $f$ on $[0, 1]$.

\subsubsection*{Theorem 5}
Let $f$ be a continuous function on an interval $I = [a, b]$. Then $f$ has the
absolute maximum value and $f$ attains it at least once in $I$. Also, $f$ has the absolute
minimum value and attains it at least once in $I$

\subsubsection*{Theorem 6}
Let $f$ be a differentiable function on a closed interval $I$ and let $c$ be any
interior point of $I$. Then
\begin{enumerate}
    \item $f '(c) = 0$ if $f$ attains its absolute maximum value at $c$
    \item $f'(c) = 0$ if $f$ attains its absolute minimum value at $c$.
\end{enumerate}

\subsubsection*{Working rule}
\begin{enumerate}
    \item Find all critical points of $f$ in the interval, i.e., find points $x$ where either
    $f ' ( x) = 0$ or $f$ is not differentiable.
    \item Take the end points of the interval
    \item At all these points (listed in Step 1 and 2), calculate the values of $f$
    \item Identify the maximum and minimum values of $f$ out of the values calculated in
    Step 3. This maximum value will be the absolute maximum (greatest) value of
    $f$ and the minimum value will be the absolute minimum (least) value of $f$ .
\end{enumerate}





\end{document}