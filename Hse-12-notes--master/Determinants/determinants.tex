\documentclass[12pt]{article}
\usepackage{graphicx} % Required for inserting images
\usepackage[margin=2cm]{geometry}
\usepackage{multicol,amsmath, amssymb}
\usepackage{xcolor}
\usepackage{titlesec}
\usepackage{pgfplots}
\usepackage{tikz}

\titleformat{\subsection}
{\color{red}\normalfont\Large\bfseries}
{\thesubsection}{1em}{}

\titleformat{\subsubsection}
{\color{blue}\normalfont\large\bfseries}
{\thesubsubsection}{1em}{}

\titlespacing{\subsection}{0pt}{0pt}{0pt} % Adjust the spacing here
\titlespacing{\subsubsection}{0pt}{\baselineskip}{0pt} % Adjust the spacing here
\newcommand{\summation}[2]{\sum\limits^{#1}_{#2}}
\usepackage{fancyhdr}
\usepackage{hyperref} % For hyperlinks

% Define the URL for the footer
\newcommand{\myURL}{https://mohammedbilalns.github.io/Math-Demystified/}

% Set up fancy headers and footers
\pagestyle{fancy}
\fancyhf{} % Clear header and footer
\rfoot{\href{\myURL}{\myURL}} % Set the right part of the footer as a hyperlink

\begin{document}
\begin{center}
    {\LARGE \textbf{Determinants} }
\end{center}

    


\subsection*{Determinant}
Determinant is a real number associated with a square matrix $A=[a_{ij}]$ of order n. The determinant of matrix A is denoted by $|A|$ or $det()$

If A=$\begin{bmatrix}
    a & b \\
    c & d
\end{bmatrix}$ then determinant of A  is denoted as $$det(A)=|A|=\begin{vmatrix} 
    a & b \\
    c & d
\end{vmatrix}$$

\subsubsection*{Determinant of matrix of order 1}
For matrix $A=\begin{bmatrix}
    a
\end{bmatrix}$ of order 1 $|A|=a$

\subsubsection*{Determinant of matrix of order 2}
Let $A=\begin{bmatrix}
    a_{11} & a_{12} \\
    a_{21} & a_{22} 
\end{bmatrix}$ be a square matrix of order 2 then  $$det(A)=|A|=\begin{vmatrix}
    a_{11} & a_{12} \\
    a_{21} & a_{22} 
\end{vmatrix} = a_{11}a_{22}-a_{12}a_{21}$$

\subsubsection*{Determinant of matrix of order 3}
Let $A=\begin{bmatrix}
    a_{11} & a_{12} & a_{13} \\
    a_{21} & a_{22} & a_{23} \\
    a_{31} & a_{32} & a_{33}
\end{bmatrix}$ be a square matrix of order 3

\textbf{Expansion along first Row ($\mathbf{R_1}$)}

$$|A|=(-1)^{1+1} a_{11} \begin{vmatrix}
    a_{22} & a_{23} \\
    a_{32} & a_{33}
 \end{vmatrix}+(-1)^{1+2}a_{12} \begin{vmatrix}
    a_{21} & a_{23} \\
    a_{31} & a_{33}
 \end{vmatrix}+(-1)^{1+3} a_{13}\begin{vmatrix}
    a_{21} & a_{22} \\
    a_{31} & a_{32}
 \end{vmatrix}$$
\textcolor{gray}{1+1, 1+2 ,1+3 are sum of positions of $a_{11},a_{12},a_{13}$ respectively, matrix of order 2 is chosen by removing row and column containing $a_{11},a_{12},a_{13}$ respectively} 

similarly we can find Determinants with other rows and columns  too

\textbf{Remarks}

\begin{itemize}
    \item For easier calculations choose the row/columns with maximum number of zeroes 
    \item If $A=kB$ then $|A|=k|B|$
    \item If $A=k^n B$ then $|A|=k^n|B|$
\end{itemize}

\subsection*{Area of a Triangle}
Area of a  Triangle with vertices $(x_1,y_1),(x_2,y_2)$ and $(x_3,y_3)$ is $$A=\frac{1}{2}\begin{vmatrix}
    x_1 & y_1 & 1 \\
    x_2 & y_2 & 1 \\
    x_3 & y_3 & 1
\end{vmatrix}$$

\subsection*{Minors and Cofactors}
\textbf{Minor} of an element $a_{ij}$ of a determinant is the determinant obtained by
deleting its ith row and jth column in which element $a_{ij}$ lies. Minor of an element $a_{ij}$ is
denoted by $M_{ij}$.

\textbf{Cofactor} of an element $a_{ij}$ , denoted by $A_{ij}$ is defined by $$A_{ij}=(-1)^{i+j}M_{ij}$$ where $M_{ij}$ is the minor of $a_{ij}$





\subsection*{Adjoint and Inverse of a matrix }
The \textbf{adjoint} of a matrix $A=[a_{ij}]_{n \times n}$ is defined as the transpose of
the matrix $A=[a_{ij}]_{n \times n}$, where $A_{ij}$ is the cofactor of the element $a_{ij}$ . Adjoint of the matrix A
is denoted by adj A $$adj \begin{bmatrix}
    a_{11} & a_{12} & a_{13} \\
    a_{21} & a_{22} & a_{23} \\
    a_{31} & a_{32} & a_{33}
\end{bmatrix}=\text{Transpose of }\begin{bmatrix}
    A_{11} & A_{12} & A_{13} \\
    A_{21} & A_{22} & A_{23} \\
    A_{31} & A_{32} & A_{33}
\end{bmatrix}$$

\begin{itemize}
    \item $adj \begin{bmatrix}
        a_{11} & a_{12} \\
        a_{21} & a_{22}
    \end{bmatrix}=\begin{bmatrix}
        a_{22} & -a_{12} \\
        -a_{21} & a_{11}
    \end{bmatrix}$
    \item \textbf{Theorem 1} - If A be any given square matrix of order n, then $$A(adj A)=(adj A) A =|A|I$$ where I is the identity matrix 
\end{itemize}


\subsubsection*{Singular Matrix}
A square matrix A is said to be \textbf{singular} if $|A| = 0$, \textbf{non singular} if $|A| \not =0$ 

\begin{itemize}
    \item \textbf{Theorem 2} - If A and B are nonsingular matrices of the same order, then AB and BA
    are also nonsingular matrices of the same order.
    \item \textbf{Theorem 3} - The determinant of the product of matrices is equal to product of their
    respective determinants, that is,$ |AB| = |A| |B|$ , where A and B are square matrices of
    the same order
    \item $|(adj A)A|= \begin{vmatrix}
        |A| & 0 & 0 \\
        0 & |A| & 0 \\
        0 & 0 & |A|
    \end{vmatrix}$
    \item \textbf{Theorem 4} - A square matrix A is invertible if and only if A is nonsingular matrix.
    \item  $A^{-1}=\frac{adj A}{|A|}$
\end{itemize}

\subsection*{Applications of Determinants and Matrices}

\subsubsection*{Consistent and inconsistent System}
A system of equations is said to be \textbf{consistent} if its solution (one
or more) exists.

A system of equations is said to be \textbf{inconsistent} if its solution
does not exist.

\subsubsection*{Solution of system of linear equations using inverse of a matrix}
Consider the system of equations
    $$a_1 x+ b_1 y +c_1 z = d_1$$
   $$a_2 x+ b_2 y +c_2 z = d_2 $$
    $$a_3 x+ b_3 y +c_3 z = d_3$$
Let $A=\begin{bmatrix}
    a_1 & b_1 & c_1 \\
    a_2 & b_2 & c_2 \\
    a_3 & b_3 & c_3 
\end{bmatrix}, X = \begin{bmatrix}
    x \\
    y \\
    z
\end{bmatrix},B=\begin{bmatrix}
    d_1 \\
    d_2 \\
    d_3
\end{bmatrix}$

Then, the system of equations can be written as, $AX = B$, i.e.,
$$A=\begin{bmatrix}
    a_1 & b_1 & c_1 \\
    a_2 & b_2 & c_2 \\
    a_3 & b_3 & c_3 
\end{bmatrix} \begin{bmatrix}
    x \\
    y \\
    z
\end{bmatrix} =\begin{bmatrix}
    d_1 \\
    d_2 \\
    d_3
\end{bmatrix}$$


\textbf{case 1} If A is a nonsingular matrix, then its inverse exists, so solution is $X=A^{-1}B$
$$\begin{bmatrix}
    x \\
    y \\
    z
\end{bmatrix}=\begin{bmatrix}
    a_1 & b_1 & c_1 \\
    a_2 & b_2 & c_2 \\
    a_3 & b_3 & c_3 
\end{bmatrix}^{-1} \begin{bmatrix}
    d_1 \\
    d_2 \\
    d_3
\end{bmatrix}$$

\textbf{case 2}If A is a singular matrix, then $| A | = 0$. then solution does not exist and the
system of equations is called inconsistent.
\end{document}
