\documentclass[12pt]{article}
\usepackage{graphicx} % Required for inserting images
\usepackage[margin=2cm]{geometry}
\usepackage{multicol,amsmath, amssymb}
\usepackage{xcolor}
\usepackage{titlesec}
\usepackage{pgfplots}
\usepackage{multirow}
\definecolor{tablecolor}{RGB}{199, 234, 251}
\titleformat{\subsection}
{\color{red}\normalfont\Large\bfseries}
{\thesubsection}{1em}{}

\titleformat{\subsubsection}
{\color{blue}\normalfont\large\bfseries}
{\thesubsubsection}{1em}{}

\titlespacing{\subsection}{0pt}{0pt}{0pt} % Adjust the spacing here
\titlespacing{\subsubsection}{0pt}{\baselineskip}{0pt} % Adjust the spacing here
\newcommand{\summation}[2]{\sum\limits^{#1}_{#2}}

\usepackage{fancyhdr}
\usepackage{hyperref} % For hyperlinks

% Define the URL for the footer
\newcommand{\myURL}{https://mohammedbilalns.github.io/Math-Demystified/}

% Set up fancy headers and footers
\pagestyle{fancy}
\fancyhf{} % Clear header and footer
\rfoot{\href{\myURL}{\myURL}} % Set the right part of the footer as a hyperlink




\begin{document}
\begin{center}
    {\LARGE \textbf{Differenial Equations} }
\end{center}
Equation involving derivative (derivatives) of the dependent variable(y)
with respect to independent variable (variables)(x) is called a differential equation. In this chapter we study the method formation of a Differential Equation and solving of a Differential Equation.

A differential equation involving derivatives of the dependent variable with respect
to only one independent variable is called an \textbf{ordinary differential equation}

eg: $$2 \frac{d^2 y}{dx^2}+ \left( \frac{dy}{dx}\right)^3 =0$$
\subsection*{Order of a differential equation}
Order of a differential equation is defined as the order of the highest order derivative of
the dependent variable with respect to the independent variable involved in the given
differential equation.

eg: $$ \frac{d^2y}{dx^2}+ y =0 (\text{\,order 2})$$ $$\frac{dy}{dx}= e^x (\text{ order 1})$$

\subsection*{Degree of a differential equation}
Degree of a DE is defined as the exponent of highest differential coefficient appearing in the equation provided the equation is made into polynomial form in all differential coefficient.
\subsection*{Homogenous DE}
A function $F(x, y)$ is said to be \textbf{homogeneous function} of degree $n$ if $F( \lambda x, \lambda y) = \lambda ^n F(x, y)$ for any nonzero constant $\lambda$. Then the function is of the form $\frac{dy}{dx}=g(\frac{y}{x})$

To solve put $y = vx $ then $$\frac{dy}{dx}=v+x \frac{dv}{dx}$$. Then we can solve the DE in the variable separable form easily 

\subsection*{First order Linear DE}
DE of the form $$\frac{dy}{dx}+Py = Q$$ where P and Q are Functions of x only 

To solve 
\begin{itemize}
    \item Find I.F = $e^{\int p dx}$
    \item Write solution as $$y (I.F) =\int (Q \times I.F)dx +C $$
\end{itemize}









\end{document}