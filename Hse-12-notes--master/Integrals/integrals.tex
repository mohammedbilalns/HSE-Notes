\documentclass[12pt]{article}
\usepackage{graphicx} % Required for inserting images
\usepackage[margin=2cm]{geometry}
\usepackage{multicol,amsmath, amssymb}
\usepackage{xcolor}
\usepackage{titlesec}
\usepackage{pgfplots}
\usepackage{multirow}
\definecolor{tablecolor}{RGB}{199, 234, 251}
\titleformat{\subsection}
{\color{red}\normalfont\Large\bfseries}
{\thesubsection}{1em}{}

\titleformat{\subsubsection}
{\color{blue}\normalfont\large\bfseries}
{\thesubsubsection}{1em}{}

\titlespacing{\subsection}{0pt}{0pt}{0pt} % Adjust the spacing here
\titlespacing{\subsubsection}{0pt}{\baselineskip}{0pt} % Adjust the spacing here
\newcommand{\summation}[2]{\sum\limits^{#1}_{#2}}

\usepackage{fancyhdr}
\usepackage{hyperref} % For hyperlinks

% Define the URL for the footer
\newcommand{\myURL}{https://mohammedbilalns.github.io/Math-Demystified/}

% Set up fancy headers and footers
\pagestyle{fancy}
\fancyhf{} % Clear header and footer
\rfoot{\href{\myURL}{\myURL}} % Set the right part of the footer as a hyperlink




\begin{document}
\begin{center}
    {\LARGE \textbf{Integrals} }
\end{center}



The original motivation for the derivative was
the problem of defining tangent lines to the graphs of
functions and calculating the slope of such lines. Integral
Calculus is motivated by the problem of defining and
calculating the area of the region bounded by the graph of
the functions.In this chapter we study different method of find indefinite integral and definite integrals of certain functions and its properties.

\subsection*{Integration}
Let $\frac{d}{dx}F(x)=f(x)$. then we write $$\int f(x) dx=F(x)+C$$  These integrals are called indefinite integrals and C is the constant of integration.


For the sake of convenience, we mention below the following symbols/terms/phrases
with their meanings as given in the Table

\begin{center}

\begingroup\setlength{\fboxsep}{0pt}
\colorbox{tablecolor}{%r}

    \begin{tabular}{ |c|c| } 
    \hline 
     \textbf{Symbols/Terms/Phrases} & \textbf{Meaning}  \\ [2.5ex]
    \hline \hline
     $\int f(x) dx$ & Integral of $f$ with respect to $x$ \\ 
     \hline
     $f(x)$ in $\int f(x) dx$ & Integrand \\ 
    \hline
    $x$ in $\int f(x) dx$ & Variable of Integration \\
    \hline
    Integrate & Find the integral \\
    \hline
    An integral of F & A function F such that $F'(x)=f(x)$ \\
    \hline
    Integration & The process of finding integral \\
    \hline
    constant of Integration & Any real number C, considered as
    constant function \\
    \hline

    \end{tabular}%
}\endgroup




    \end{center}

\subsubsection*{Some properties}
    \begin{itemize}
        \item Indefinite integral is a collection of family of curves, each of which is obtained by translating one of the curves parallel to itself upward or downwards along the y-axis.
        \item $\int f(x) \pm g(x)=\int f(x) \pm \int g(x)$
        \item For any real number $k$,$\int k f(x) =k \int f(x)$
    \end{itemize}

    \subsection*{Some Standard Results}
    \includegraphics*[scale=0.67]{1.png}

    \includegraphics*[scale=0.67]{2.png}

    \subsection*{Methods of Integration}
    \subsubsection*{Integration by substitution}
    The given integral $I = \int f(x)dx$ is transformed into another form by changing the independent variable $x$ to $t$ by substituting $x = g(t)$  So that $\frac{dx}{dt} = g'(t) \Rightarrow dx = g'(t)dt$ $$I=\int f(x) dx =\int f(g(t)) g'(t)dt$$

    \subsubsection*{Some more Standard Results derived using substitution}
    \begin{itemize}
        \item $\int tan \,x \, dx =log|sec \, x| +C$
        \item $\int cot \, x \, dx = log |sin \, x | + C $
        \item $\int sec \, x \, dx = log |sec \, x + tan \, x| + C $
        \item $\int cosec \, x \, dx = log |cosec \, x - cot \, x| + C $
    \end{itemize}

    \subsection*{Integrals of some particular Functions}
    \begin{enumerate}
        \item $\int \frac{dx}{x^2-a^2}=\frac{1}{2a} log|\frac{x-a}{x+a}|+C$
        \item  $\int \frac{dx}{a^2-x^2}=\frac{1}{2a} log|\frac{a+x}{a-x}|+C$
        \item $\int \frac{dx}{x^2+a^2}=\frac{1}{a} tan^{-1}\frac{x}{a}+C$
        \item $\int \frac{dx}{\sqrt{x^2-a^2}}=log|x+\sqrt{x^2-a^2}|+C$
        \item $\int \frac{dx}{\sqrt{x^2+a^2}}=log|x+\sqrt{x^2+a^2}+C|$
        \item $\int \frac{dx}{\sqrt{a^2-x^2}}= sin^{-1}\frac{x}{a}+C$
        \item \textbf{To find the integral of the type $\frac{px+q}{ax^2+bx+c}$ and $\frac{px+q}{\sqrt{ax^2+bx+c}}$ } where p , q , a , b , c are constants , we are to find real numbers A, B such that $$px+q=A\frac{d}{dx}(ax^2+bx+c)+B= A (2ax+b)+B$$ To determine A and B, we equate from both sides the coefficients of x and the
        constant terms. A and B are thus obtained and hence the integral is reduced to
        one of the known forms.
    \end{enumerate}

    \subsection*{Integration by Partial Fractions}
    Consider integrals of the form  $\int \frac{P(x)}{Q(x)} dx $ whre P and Q are polinomials in x and $Q(x) \not = 0$. If the degree of $P(x)$ is less than $Q(x)$, then the rational function is \textbf{Proper rational function} otherwise it is called \textbf{Improper rational function}


    If $\frac{P(x)}{Q(x)}$ is improper function, first it should be converted to proper by long division and now it takes the form  $\frac{P(x)}{Q(x)}=T(x)+\frac{P_1(x)}{Q(x)}$    Where T(x) is polynomial in x and    $\frac{P_1(x)}{Q(x)}$     is a proper function.

    Now if $\frac{P(x)}{Q(x)}$     is proper function we factorise the denominator Q(x) into simpler polynomials and decompose into simpler rational function. For this we use the following table.

    \begin{center}
        \includegraphics*[scale= 0.5]{3.png}

        
    \end{center}

    \subsection*{Integration by Parts}
    $$\int f(x) g(x) dx = f(x) \int g(x) dx - \int \left( f'(x) \left[ \int g(x) dx \right] \right) dx $$

    Here the priority of taking first function and second function is more important, for this use order of the letters in words ILATE, where

    \begin{itemize}
        \item I- Inverse Triganometric function
        \item L - Logarithmic function 
        \item A - Algebraic function 
        \item T - Triganometric function 
        \item E - Exponential function 
    \end{itemize}
    \subsubsection*{A Special case}
    $$\int e^x [f(x)+f'(x)] dx = e^x f(x)+C$$


    \subsection*{Integrals of some more types}
    \begin{itemize}
        \item $\int \sqrt{x^2-a^2} dx =\frac{1}{2} x \sqrt{x^2-a^2} -\frac{a^2}{2} log|x+\sqrt{x^2-a^2}|+c$
        \item $\int \sqrt{x^2+a^2} dx =\frac{1}{2} x \sqrt{x^2+a^2} +\frac{a^2}{2} log|x+\sqrt{x^2+a^2}|+c$
        \item $\int \sqrt{a^2-x^2}dx = \frac{1}{2} x \sqrt{a^2-x^2}+\frac{a^2}{2}sin^{-1}\frac{x}{a}+c$

    \end{itemize}

    \subsection*{Definite Integral}
    A definite integral has a unique value. A definite integral is denoted by $\int_a^b f(x) dx$, where a is the upper limit and b is the lower limit of the integral. If $\frac{d}{dx} F(x)=f(x)$ and $\int f(x)dx = F(x) + C$ , then $$ \int_a^b f(x) dx= F(a)-F(b)$$


    \subsection*{Fundamental Theorem of Calculus}
    We have defined $\int_a^b f(x) dx$ as the area of  the region bounded by the curve y = f (x),
the ordinates $x = a$ and $x = b$ and x-axis.We denote this
function of x by A(x).$$A(x)=\int_a^b f(x) dx$$

\subsubsection*{First fundamental theorem of integral calculus}
Let f be a continuous function on the closed interval [a, b] and let A (x) be
the area function. Then $A'(x) = f (x)$, for all $x \in [a, b].$

\subsubsection*{Second fundamental theorem of integral calculus}
Let f be continuous function defined on the closed interval [a, b] and F be
an anti derivative of f. Then $$\int_a^b f(x) dx = [F(x)]_b^a =F(b)-F(a) $$


\subsection*{Some Properties of Definite Integrals}
\begin{center}
    \includegraphics*[scale=0.5]{41.png}
\end{center}
\end{document}