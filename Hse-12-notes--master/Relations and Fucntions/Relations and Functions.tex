\documentclass[11pt]{article}
\usepackage{graphicx} % Required for inserting images
\usepackage[margin=2cm]{geometry}
\usepackage{multicol,amsmath, amssymb}
\usepackage{xcolor}
\usepackage{titlesec}
\usepackage{pgfplots}
\usepackage[scaled=1.12]{nimbusmononarrow}% typewriter font
\usepackage[sfdefault,subscriptcorrection]{notomath} 
\newcommand{\notinsyllabus}[1]{
    \begin{center}
        \fcolorbox{red}{yellow!20}{\parbox{0.9\linewidth}{
            \textbf{Not in latest Syllabus:} #1
        }}
    \end{center}
}
\titleformat{\subsection}
{\color{red}\normalfont\Large\bfseries}
{\thesubsection}{1em}{}

\titleformat{\subsubsection}
{\color{blue}\normalfont\large\bfseries}
{\thesubsubsection}{1em}{}

\titlespacing{\subsection}{0pt}{0pt}{0pt} % Adjust the spacing here
\titlespacing{\subsubsection}{0pt}{\baselineskip}{0pt} % Adjust the spacing here
\newcommand{\summation}[2]{\sum\limits^{#1}_{#2}}
\usepackage{fancyhdr}
\usepackage{hyperref} % For hyperlinks

% Define the URL for the footer
\newcommand{\myURL}{https://mohammedbilalns.github.io/Math-Demystified/}

% Set up fancy headers and footers
\pagestyle{fancy}
\fancyhf{} % Clear header and footer
\rfoot{\href{\myURL}{\myURL}} % Set the right part of the footer as a hyperlink

\begin{document}
\begin{center}
    {\LARGE \textbf{Relations and functions} }
\end{center}

\begin{multicols}{2}

    A relation from a non-empty set A to a non-empty set B is a subset of A × B.
    
    In this chapter we study different types of relations and functions, composition of functions, and binary operations.
    \subsubsection*{Examples}
    \begin{itemize}
        \item $\{(a,b) \in A \times B: \text{a is brother of b}\}$
		\item $\{(a,b) \in A \times B:$ age of a is greater than age of b\}
    \end{itemize}

    \subsection*{Types of Relations }
    \begin{itemize}
        \item \textbf{Empty Relation:} $R:A \rightarrow A$ given by $R=\phi \subset A \times A$
        \item \textbf{Universal Relation} $R:A \rightarrow A$ given by $R=A \times A$
        \item \textbf{Reflexive Relation} $R:A \rightarrow A$ with $(a,a) \in R, \forall a \in A$
        \item \textbf{Symmetric Relation} $R:A \rightarrow A$ with $(a,b) \in R \Rightarrow (b,a) \in R, a,b \in A$
        \item \textbf{Transitive Relation} $R:A \rightarrow A$ with $(a,b) \in R $ and $(b,c) \in R \Rightarrow (a,c) \in R$
        \item \textbf{Equivalence Relation} Relation which is Reflexive , Symmetric and Transitive.
    \end{itemize}
    \subsubsection*{Equivalence Class}
    Let A be an Equivalence Relation in a set A. If $a \in A$, then the subset $\{x \in A ,(x,a) \in R \}$ of A is called the Equivalence class corresponding to 'a' and is denoted by $[a]$.

\subsection*{Types of functions}
\subsubsection*{One-One or Injective function.}
A function $f : A \rightarrow B$ is said to be \emph{One-One} or \emph{Injective}, if the image of distinct elements of A under $f$ are distinct. 

i.e;$ f(x_1) = f(x_2) \Rightarrow x_1 = x_2$.

Otherwise, f is called many-one.

If lines parallel to x-axis meet the curve at two or more points, then the function is not one-one.

\subsubsection*{Onto or Surjective function}
A function $f : A \rightarrow B$ is said to be \emph{Onto} or \emph{Surjective}, if every element of $B$ is some image of some elements of $A$ under $f$.

ie; If for every element $y \in Y$ then there exists an element $x$ in $A$ such that $f(x) = y$.
\subsubsection*{Bijective Functions}
A function $f : A \rightarrow B$ is said to be Bijective if it is both One-One and Onto.
\notinsyllabus{
\subsection*{Composition of functions and Invertible function}
\subsubsection*{Composition of Functions.}
Let $f : A \rightarrow B$ and $g : B \rightarrow C$ be two functions. Then the composition of $f$ and $g$ denoted by is $gof$ defined
$gof : A \rightarrow C$ and $gof (x) = g(f(x))$.

\begin{itemize}
    \item If $f : A \rightarrow B$ and $g : B \rightarrow C$ are One-One, then $gof : A \rightarrow C$ is One-One.
    \item If $f : A \rightarrow B$ and $g : B \rightarrow C$ are Onto, then $gof : A \rightarrow C$ is Onto.gof : A → C
    \item If $f : A \rightarrow B$ and $g : B \rightarrow C$ are Bijective,  $ \iff gof : A \rightarrow C$ is Bijective.
\end{itemize}

\subsection*{Inverse of a  Function}
If $f : A \rightarrow B$ is defined to be invertible, if there exists a function $g : B \rightarrow A$ such that $gof = I_A$ and
$fog = I_B$. The function $g$ is called the inverse of $'f'$ and is denoted by $f^{-1}$.

\begin{itemize}
    \item If function $f : A \rightarrow B$ is invertible only if $f$ is bijective.
    \item $(gof)^{-1} = f^{-1}og^{-1}$.
\end{itemize}

}
\end{multicols}

\end{document}
